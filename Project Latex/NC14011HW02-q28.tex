سوال تئوری ۱۷

از آن‌جایی که در کل دو خوشه سلایق داریم و هرکس در یکی از این دو است، به این نتیجه می‌رسیم که افراد هم خوشه، ستون‌ها و سطرهای یکسانی در ماتریس مجاورت  خواهند داشت. پس کل ستون‌ها به دو نوع تقسیم می‌شوند. به همین دلیل تعداد ستون‌های مستقل خطی دوتا است و می‌دانیم تعداد مقادیر ویژه غیر صفر برابر با رنک ماتریس است در نتیجه دو مقدار ویژه غیرصفر داریم.
می‌دانیم هر ماتریس متقارن، n بردار ویژه عمود به هم دارد. پس با جای‌گذاری مقادیر ویژه در معادله اولیه ،n بردار ویژه بدست خواهد آمد.

می دانیم ماتریس W به شکل ماتریس یک گراف منتظم است و برای این ماتریس‌ها، بردار 1، یک بردار ویژه است، پس داریم:
(فرض کنیم m تا راس از یک دسته و باقی از دسته دیگر باشند.)

$$W 1 = \lambda 1 \rightarrow mp + (n-m)q = \lambda $$
\\
$$ trace(W) = \lambda_{1} + \lambda_{2} = np $$
\\
$$ \rightarrow \lambda_{1} = mp + (n-m)q, \space \lambda_{2} = (n-m)p + (m-n)q $$

جمع درایه های روی قطر ماتریس برابر با جمع مقادیر ویژه آن است.

بردارهای ویژه نیز با جایگذاری مقادیر ویژه بدست می‌آیند.
