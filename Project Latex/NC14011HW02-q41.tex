سوال تئوری ۲۵

هر جفت مستقل از بقیه به احتمال
$p$
یال می‌کشیم و به احتمال 
$$
q=1-p
$$
یال نمی‌کشیم.
برای 
$m$
تا از جفت ها در صورت یال کشیدن رابطه درست تعیین شده است.
یعنی به احتمال 
$p$


برای 
$k=\binom{n}{2}-m$
جفت هم به احتمال 
$q$
رابطه هم‌سلیقگی درست تعیین می‌شود.

$X$
را تعداد یال‌هایی تعریف می‌کنیم که به درستی کشیده شده‌اند و 
$Y$
را هم تعداد یال‌هایی تعریف می‌کنیم که به درستی کشیده نشده‌اند.
هدف سوال
$P(X+Y \ge \frac{m+k}{5})$
است.


می‌دانیم که:

$$X \sim Binom(m, p)$$

$$Y \sim Binom(k, q)$$

$$E[X]=mp \qquad Var[x]=mpq$$

$$E[Y]=kq \qquad Var[x]=kqp$$

جواب را می‌توان به صورت جمع یک سری نوشت ولی برای محاسبه رابطه مستقیم،
با قضیه حد مرکزی 
$X$
و
$Y$
را تخمین می‌زنیم.

$$X \approx Normal(mp, mpq)$$

$$Y \approx Normal(kq, kpq)$$

و جمع دو متغیر تصادفی که از توزیع نورمال پیروی می‌کنند، یک توزیع نورمال است.

$$X+Y \approx Normal(mp+kq, mpq+kpq = (m+k)pq )$$

$$Z=\frac{X+Y-mp-kq}{\sqrt{(m+k)pq}} \qquad Z \approx Normal(0, 1)$$

$$P(X+Y \ge \frac{m+k}{5}) \ = \ P(Z \ge \frac{\frac{m+k}{5}-mp-kq}{\sqrt{(m+k)pq}}) 
    = \ 1-\phi(\frac{\frac{m+k}{5}-mp-kq}{\sqrt{(m+k)pq}})$$