سوال تئوری ۶

این روش در اصل همان روش gradient descend است که کاربرد زیادی در محاسبات عددی دارد. در این روش از نقطه یا نقاطی روی نمودار شروع می‌کنیم و در هر گام مقدار اندکی در راستای گرادیان در آن نقطه جابجا می‌شویم و به نقطه‌ي جدید می‌رویم. اگر این کار را با تعداد گام و طول گام مناسب انجام بدهیم انتظار داریم که به نقطه‌ی اکسترمم موضعی برسیم. 
اما برای این کار شرایطی لازم است. اولا اینکه تابع پوسته باشد که تابع ما جمع چند لگاریتم پیوسته است پس پیوسته است. همچنین نوسانات شدید در تابع باعث می‌شوند که روش به خوبی کار نکند. در این تابع ما نوسانات شدید هم وجود ندارد. هم‌چنین نکته‌ی دیگر این است که این روش به ما اکسترمم موضعی را می‌دهد و گلوبال و به شرطی که فقط یک اکسترمم داشته داشته باشیم قطعا به اکسترمم گلوبال می‌رسیم. همچنین نواحی با گرادیان صفر نباید وجود داشته باشند تا الگوریتم متوقف نشود و یا در دور نیافتد.