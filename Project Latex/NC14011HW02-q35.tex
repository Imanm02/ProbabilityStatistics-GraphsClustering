سوال تئوری ۲۲

خلاصه‌ی تحقیق و گزارش رو این‌جا می‌نویسیم. یکی از منابع این تحقیق، همان فایل
$Kmeans$
داده شده هستش.

در واقع این الگوریتم برای خوشه‌بندی داده‌هایی به کار می‌رود که به شکل پارتیشن‌بندی قرار ندارند. یعنی داده‌هایی که در یک خوشه نزدیک به هم هستند. سپس داده‌ها تبدیل می‌شوند و در فضایی
$n$
بعدی می‌روند و خوشه‌ها آن‌جا پارتیشن‌بندی می‌شوند.
در این‌جا می‌توان با الگوریتمی مثل
$k-means$
داده‌ها را به خوشه و پارتیشن‌های مختلف افراز کرد.